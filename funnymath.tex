\documentclass[a4paper]{minimal}
\usepackage[english,russian]{babel}
\usepackage[utf8]{inputenc}
\usepackage{amsmath}
\usepackage{tikz}

\begin{document}

\newcount\dv
\newcount\res

%% Calculates the #1 % 4
\newcommand{\calcmod}[1]{
\dv=#1
\res=\the\dv
\divide\dv by 4
\multiply\dv by -4
\advance\res by \the\dv
}

Программа основана на использовании формулы окружности $R = \sqrt{x^2+y^2},$ но результат работы этой программы показывает, что окружности сливаются и дают в результате устойчивые
изображения квадратов.

График ниже получен c использованием следующих итерационных формул, предложенных для построения узоров Б. Мартином из Астонского университета в Бирмингеме:

$$
\left\{\begin{aligned}
x_i  = & y_{i-1} - sign(x_{i-1})\cdot\sqrt{\left|b\cdot x_{i-1} - c\right|},&x_0 = 0\\
y_i  = & a - x_{i-1},&y_0 = 0
\end{aligned}\right.
$$

При значениях констант:

$$
a = 40, b = 20, c = 100.
$$

\begin{tikzpicture}[x=1mm,y=1mm,scale=.35]
\pgfmathsetmacro{\constA}{40}%
\pgfmathsetmacro{\constB}{20}%
\pgfmathsetmacro{\constC}{100}%
\foreach[%
	remember=\x as \xx (initially 0),
        remember=\y as \yy (initially 0),
        evaluate=\x using (\xx>0) ? \yy - sqrt(abs(\constB*\xx-\constC)) : \yy + sqrt(abs(\constB*\xx-\constC)),
        evaluate=\y using \constA - \xx,
        ] \i in {0,...,11000} {%
	\calcmod{\i}
	\ifnum\res=0 \fill[color=red](\x,\y) circle(1pt);
	\fi
	\ifnum\res=1 \fill[color=yellow](\x,\y) circle(1pt);
	\fi
	\ifnum\res=2 \fill[color=blue](\x,\y) circle(1pt);
	\fi
	\ifnum\res=3 \fill[color=black](\x,\y) circle(1pt);
	\fi
}%
\end{tikzpicture}

А вот эти формулы напечатаны в школьной книге с ошибкой (рисуют овал по крайней мере при $A=60$):

$$
\left\{\begin{aligned}
x_i  = & y_{i-1} - \sin\left(x_{i-1}\right),&x_0 = 0\\
y_i  = & a - x_{i-1},&y_0 = 0
\end{aligned}\right.
$$

\end{document}
